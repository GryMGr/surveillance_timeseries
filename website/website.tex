\documentclass[]{book}
\usepackage{lmodern}
\usepackage{amssymb,amsmath}
\usepackage{ifxetex,ifluatex}
\usepackage{fixltx2e} % provides \textsubscript
\ifnum 0\ifxetex 1\fi\ifluatex 1\fi=0 % if pdftex
  \usepackage[T1]{fontenc}
  \usepackage[utf8]{inputenc}
\else % if luatex or xelatex
  \ifxetex
    \usepackage{mathspec}
  \else
    \usepackage{fontspec}
  \fi
  \defaultfontfeatures{Ligatures=TeX,Scale=MatchLowercase}
\fi
% use upquote if available, for straight quotes in verbatim environments
\IfFileExists{upquote.sty}{\usepackage{upquote}}{}
% use microtype if available
\IfFileExists{microtype.sty}{%
\usepackage{microtype}
\UseMicrotypeSet[protrusion]{basicmath} % disable protrusion for tt fonts
}{}
\usepackage[margin=1in]{geometry}
\usepackage{hyperref}
\hypersetup{unicode=true,
            pdftitle={Time Series and Longitudinal Analysis},
            pdfauthor={Richard White},
            pdfborder={0 0 0},
            breaklinks=true}
\urlstyle{same}  % don't use monospace font for urls
\usepackage{natbib}
\bibliographystyle{apalike}
\usepackage{color}
\usepackage{fancyvrb}
\newcommand{\VerbBar}{|}
\newcommand{\VERB}{\Verb[commandchars=\\\{\}]}
\DefineVerbatimEnvironment{Highlighting}{Verbatim}{commandchars=\\\{\}}
% Add ',fontsize=\small' for more characters per line
\usepackage{framed}
\definecolor{shadecolor}{RGB}{248,248,248}
\newenvironment{Shaded}{\begin{snugshade}}{\end{snugshade}}
\newcommand{\KeywordTok}[1]{\textcolor[rgb]{0.13,0.29,0.53}{\textbf{#1}}}
\newcommand{\DataTypeTok}[1]{\textcolor[rgb]{0.13,0.29,0.53}{#1}}
\newcommand{\DecValTok}[1]{\textcolor[rgb]{0.00,0.00,0.81}{#1}}
\newcommand{\BaseNTok}[1]{\textcolor[rgb]{0.00,0.00,0.81}{#1}}
\newcommand{\FloatTok}[1]{\textcolor[rgb]{0.00,0.00,0.81}{#1}}
\newcommand{\ConstantTok}[1]{\textcolor[rgb]{0.00,0.00,0.00}{#1}}
\newcommand{\CharTok}[1]{\textcolor[rgb]{0.31,0.60,0.02}{#1}}
\newcommand{\SpecialCharTok}[1]{\textcolor[rgb]{0.00,0.00,0.00}{#1}}
\newcommand{\StringTok}[1]{\textcolor[rgb]{0.31,0.60,0.02}{#1}}
\newcommand{\VerbatimStringTok}[1]{\textcolor[rgb]{0.31,0.60,0.02}{#1}}
\newcommand{\SpecialStringTok}[1]{\textcolor[rgb]{0.31,0.60,0.02}{#1}}
\newcommand{\ImportTok}[1]{#1}
\newcommand{\CommentTok}[1]{\textcolor[rgb]{0.56,0.35,0.01}{\textit{#1}}}
\newcommand{\DocumentationTok}[1]{\textcolor[rgb]{0.56,0.35,0.01}{\textbf{\textit{#1}}}}
\newcommand{\AnnotationTok}[1]{\textcolor[rgb]{0.56,0.35,0.01}{\textbf{\textit{#1}}}}
\newcommand{\CommentVarTok}[1]{\textcolor[rgb]{0.56,0.35,0.01}{\textbf{\textit{#1}}}}
\newcommand{\OtherTok}[1]{\textcolor[rgb]{0.56,0.35,0.01}{#1}}
\newcommand{\FunctionTok}[1]{\textcolor[rgb]{0.00,0.00,0.00}{#1}}
\newcommand{\VariableTok}[1]{\textcolor[rgb]{0.00,0.00,0.00}{#1}}
\newcommand{\ControlFlowTok}[1]{\textcolor[rgb]{0.13,0.29,0.53}{\textbf{#1}}}
\newcommand{\OperatorTok}[1]{\textcolor[rgb]{0.81,0.36,0.00}{\textbf{#1}}}
\newcommand{\BuiltInTok}[1]{#1}
\newcommand{\ExtensionTok}[1]{#1}
\newcommand{\PreprocessorTok}[1]{\textcolor[rgb]{0.56,0.35,0.01}{\textit{#1}}}
\newcommand{\AttributeTok}[1]{\textcolor[rgb]{0.77,0.63,0.00}{#1}}
\newcommand{\RegionMarkerTok}[1]{#1}
\newcommand{\InformationTok}[1]{\textcolor[rgb]{0.56,0.35,0.01}{\textbf{\textit{#1}}}}
\newcommand{\WarningTok}[1]{\textcolor[rgb]{0.56,0.35,0.01}{\textbf{\textit{#1}}}}
\newcommand{\AlertTok}[1]{\textcolor[rgb]{0.94,0.16,0.16}{#1}}
\newcommand{\ErrorTok}[1]{\textcolor[rgb]{0.64,0.00,0.00}{\textbf{#1}}}
\newcommand{\NormalTok}[1]{#1}
\usepackage{longtable,booktabs}
\usepackage{graphicx,grffile}
\makeatletter
\def\maxwidth{\ifdim\Gin@nat@width>\linewidth\linewidth\else\Gin@nat@width\fi}
\def\maxheight{\ifdim\Gin@nat@height>\textheight\textheight\else\Gin@nat@height\fi}
\makeatother
% Scale images if necessary, so that they will not overflow the page
% margins by default, and it is still possible to overwrite the defaults
% using explicit options in \includegraphics[width, height, ...]{}
\setkeys{Gin}{width=\maxwidth,height=\maxheight,keepaspectratio}
\IfFileExists{parskip.sty}{%
\usepackage{parskip}
}{% else
\setlength{\parindent}{0pt}
\setlength{\parskip}{6pt plus 2pt minus 1pt}
}
\setlength{\emergencystretch}{3em}  % prevent overfull lines
\providecommand{\tightlist}{%
  \setlength{\itemsep}{0pt}\setlength{\parskip}{0pt}}
\setcounter{secnumdepth}{5}
% Redefines (sub)paragraphs to behave more like sections
\ifx\paragraph\undefined\else
\let\oldparagraph\paragraph
\renewcommand{\paragraph}[1]{\oldparagraph{#1}\mbox{}}
\fi
\ifx\subparagraph\undefined\else
\let\oldsubparagraph\subparagraph
\renewcommand{\subparagraph}[1]{\oldsubparagraph{#1}\mbox{}}
\fi

%%% Use protect on footnotes to avoid problems with footnotes in titles
\let\rmarkdownfootnote\footnote%
\def\footnote{\protect\rmarkdownfootnote}

%%% Change title format to be more compact
\usepackage{titling}

% Create subtitle command for use in maketitle
\newcommand{\subtitle}[1]{
  \posttitle{
    \begin{center}\large#1\end{center}
    }
}

\setlength{\droptitle}{-2em}

  \title{Time Series and Longitudinal Analysis}
    \pretitle{\vspace{\droptitle}\centering\huge}
  \posttitle{\par}
    \author{Richard White}
    \preauthor{\centering\large\emph}
  \postauthor{\par}
      \predate{\centering\large\emph}
  \postdate{\par}
    \date{2018-11-02}

\usepackage{booktabs}

\begin{document}
\maketitle

{
\setcounter{tocdepth}{1}
\tableofcontents
}
\chapter*{Preface}\label{preface}
\addcontentsline{toc}{chapter}{Preface}

When dealing with data measured over time, there are two kinds of
analyses that can be performed.

``Time series'' analyses generally deal with one variable (the outcome).
We can then either:

\begin{enumerate}
\def\labelenumi{\arabic{enumi}.}
\tightlist
\item
  Predict the future only using the previous observations. E.g. predict
  tomorrow's temperature, using today's and yesterday's temperature as
  exposures. We will not be focusing on these kinds of analyses.
\item
  Estimate descriptive statistics about the data. E.g. Today's data is
  much higher than expected (outbreak?). We will focus on these kinds of
  analyses.
\end{enumerate}

If we have more than one variable measured over time (e.g.~outcome and
an exposure) then we can run regression analyses. E.g. seeing how the
number of tuberculosis patients (outcome) is affected by the number of
immigrants to Norway (exposure) over a 20 year period. We will focus on
these kinds of analyses.

It is important to note that if we define our exposure as ``time'' then
we can use the regression framework to estimate descriptive statistics
about the data. This means we can use the same regression framework for
the two kinds of analyses we will be focusing on.

The ``regression framework'' is very similar to ordinary regressions
that you have been working with for many years. The only difference is
that some of the data \textbf{may} have more advanced data structures
that your normal methods cannot handle.

\chapter{Definitions and Scenarios}\label{definitions-and-scenarios}

\section{Panel Data}\label{panel-data}

Panel data is a set of data with measurements repeated at equally spaced
points. For example, number of influenza cases recorded every day, or
every week, or every year would be considered panel data. The number of
influenza cases on Jan 31, Feb 3, and Nov 21 in 2018 would not be
considered panel data.

\section{Autocorrelation}\label{autocorrelation}

When you have panel data, autocorrelation is the correlation between
subsequent observations. For example, if you have daily observations,
then the 1 day autocorrelation is the correlation between observations 1
day apart, and likewise the 2 day autocorrelation is the correlation
between observations 2 days apart.

\section{Scenarios}\label{scenarios}

In this course we will consider three scenarios where we have multiple
observations for each geographical area:

\begin{itemize}
\tightlist
\item
  Panel data: One geographical area, with/without autocorrelation
\item
  Not panel data: Multiple geographical areas
\item
  Panel data: Multiple geographical areas, with/without autocorrelation
\end{itemize}

Note, the following scenario can be covered by standard regression
models:

\begin{itemize}
\tightlist
\item
  Multiple geographical areas, one time point/observation per
  geographical area
\end{itemize}

\section{Useful Code}\label{useful-code}

This code is used to calculate prediction intervals. In its most basic
form it is:

\[
\text{95% CI} = \text{sample average} \pm 1.96 \times \text{sample standard deviation} \sqrt{ 1 + 1 / n}
\] However, due to the skewness of the count data, we often choose to
use a \texttt{2/3s\ transformation}.

\begin{Shaded}
\begin{Highlighting}[]
\NormalTok{FarringtonThreshold <-}\StringTok{ }\ControlFlowTok{function}\NormalTok{(pred, phi, }\DataTypeTok{alpha =} \OtherTok{NULL}\NormalTok{, }\DataTypeTok{z =} \OtherTok{NULL}\NormalTok{, }\DataTypeTok{skewness.transform =} \StringTok{"none"}\NormalTok{) \{}
\NormalTok{  mu0 <-}\StringTok{ }\NormalTok{pred}\OperatorTok{$}\NormalTok{fit}
\NormalTok{  tau <-}\StringTok{ }\NormalTok{phi }\OperatorTok{+}\StringTok{ }\NormalTok{(pred}\OperatorTok{$}\NormalTok{se.fit}\OperatorTok{^}\DecValTok{2}\NormalTok{) }\OperatorTok{/}\StringTok{ }\NormalTok{mu0}
  \ControlFlowTok{switch}\NormalTok{(skewness.transform, }\DataTypeTok{none =}\NormalTok{ \{}
\NormalTok{    se <-}\StringTok{ }\KeywordTok{sqrt}\NormalTok{(mu0 }\OperatorTok{*}\StringTok{ }\NormalTok{tau)}
\NormalTok{    exponent <-}\StringTok{ }\DecValTok{1}
\NormalTok{  \}, }\StringTok{`}\DataTypeTok{1/2}\StringTok{`}\NormalTok{ =}\StringTok{ }\NormalTok{\{}
\NormalTok{    se <-}\StringTok{ }\KeywordTok{sqrt}\NormalTok{(}\DecValTok{1} \OperatorTok{/}\StringTok{ }\DecValTok{4} \OperatorTok{*}\StringTok{ }\NormalTok{tau)}
\NormalTok{    exponent <-}\StringTok{ }\DecValTok{1} \OperatorTok{/}\StringTok{ }\DecValTok{2}
\NormalTok{  \}, }\StringTok{`}\DataTypeTok{2/3}\StringTok{`}\NormalTok{ =}\StringTok{ }\NormalTok{\{}
\NormalTok{    se <-}\StringTok{ }\KeywordTok{sqrt}\NormalTok{(}\DecValTok{4} \OperatorTok{/}\StringTok{ }\DecValTok{9} \OperatorTok{*}\StringTok{ }\NormalTok{mu0}\OperatorTok{^}\NormalTok{(}\DecValTok{1} \OperatorTok{/}\StringTok{ }\DecValTok{3}\NormalTok{) }\OperatorTok{*}\StringTok{ }\NormalTok{tau)}
\NormalTok{    exponent <-}\StringTok{ }\DecValTok{2} \OperatorTok{/}\StringTok{ }\DecValTok{3}
\NormalTok{  \}, \{}
    \KeywordTok{stop}\NormalTok{(}\StringTok{"No proper exponent in algo.farrington.threshold."}\NormalTok{)}
\NormalTok{  \})}
  \ControlFlowTok{if}\NormalTok{ (}\KeywordTok{is.null}\NormalTok{(z)) z <-}\StringTok{ }\KeywordTok{qnorm}\NormalTok{(}\DecValTok{1} \OperatorTok{-}\StringTok{ }\NormalTok{alpha }\OperatorTok{/}\StringTok{ }\DecValTok{2}\NormalTok{)}
\NormalTok{  lu <-}\StringTok{ }\NormalTok{(mu0}\OperatorTok{^}\NormalTok{exponent }\OperatorTok{+}\StringTok{ }\NormalTok{z }\OperatorTok{*}
\StringTok{    }\NormalTok{se)}\OperatorTok{^}\NormalTok{(}\DecValTok{1} \OperatorTok{/}\StringTok{ }\NormalTok{exponent)}

  \KeywordTok{return}\NormalTok{(lu)}
\NormalTok{\}}
\end{Highlighting}
\end{Shaded}

\chapter{Panel Data - One Area}\label{panel-data---one-area}

\section{Aim}\label{aim}

We are given a dataset containing daily counts of diseases from one
geographical area. We want to identify:

\begin{itemize}
\tightlist
\item
  Does seasonality exist?
\item
  If seasonality exists, when are the high/low seasons?
\item
  Is there a general yearly trend (i.e.~increasing or decreasing from
  year to year?)
\item
  Is daily rainfall associated with the number of cases?
\end{itemize}

\begin{Shaded}
\begin{Highlighting}[]
\KeywordTok{library}\NormalTok{(data.table)}
\KeywordTok{library}\NormalTok{(ggplot2)}
\KeywordTok{set.seed}\NormalTok{(}\DecValTok{4}\NormalTok{)}

\NormalTok{AMPLITUDE <-}\StringTok{ }\FloatTok{1.5}
\NormalTok{SEASONAL_HORIZONTAL_SHIFT <-}\StringTok{ }\DecValTok{20}

\NormalTok{d <-}\StringTok{ }\KeywordTok{data.table}\NormalTok{(}\DataTypeTok{date=}\KeywordTok{seq.Date}\NormalTok{(}
  \DataTypeTok{from=}\KeywordTok{as.Date}\NormalTok{(}\StringTok{"2000-01-01"}\NormalTok{),}
  \DataTypeTok{to=}\KeywordTok{as.Date}\NormalTok{(}\StringTok{"2018-12-31"}\NormalTok{),}
  \DataTypeTok{by=}\DecValTok{1}\NormalTok{))}
\NormalTok{d[,date}\OperatorTok{:}\ErrorTok{=}\KeywordTok{as.Date}\NormalTok{(date,}\DataTypeTok{origin=}\StringTok{"1970-01-1"}\NormalTok{)]}
\NormalTok{d[,year}\OperatorTok{:}\ErrorTok{=}\KeywordTok{as.numeric}\NormalTok{(}\KeywordTok{format.Date}\NormalTok{(date,}\StringTok{"%G"}\NormalTok{))]}
\NormalTok{d[,week}\OperatorTok{:}\ErrorTok{=}\KeywordTok{as.numeric}\NormalTok{(}\KeywordTok{format.Date}\NormalTok{(date,}\StringTok{"%V"}\NormalTok{))]}
\NormalTok{d[,month}\OperatorTok{:}\ErrorTok{=}\KeywordTok{as.numeric}\NormalTok{(}\KeywordTok{format.Date}\NormalTok{(date,}\StringTok{"%m"}\NormalTok{))]}
\NormalTok{d[,yearMinus2000}\OperatorTok{:}\ErrorTok{=}\NormalTok{year}\OperatorTok{-}\DecValTok{2000}\NormalTok{]}
\NormalTok{d[,dailyrainfall}\OperatorTok{:}\ErrorTok{=}\KeywordTok{runif}\NormalTok{(.N, }\DataTypeTok{min=}\DecValTok{0}\NormalTok{, }\DataTypeTok{max=}\DecValTok{10}\NormalTok{)]}

\NormalTok{d[,dayOfYear}\OperatorTok{:}\ErrorTok{=}\KeywordTok{as.numeric}\NormalTok{(}\KeywordTok{format.Date}\NormalTok{(date,}\StringTok{"%j"}\NormalTok{))]}
\NormalTok{d[,seasonalEffect}\OperatorTok{:}\ErrorTok{=}\KeywordTok{sin}\NormalTok{(}\DecValTok{2}\OperatorTok{*}\NormalTok{pi}\OperatorTok{*}\NormalTok{(dayOfYear}\OperatorTok{-}\NormalTok{SEASONAL_HORIZONTAL_SHIFT)}\OperatorTok{/}\DecValTok{365}\NormalTok{)]}
\NormalTok{d[,mu }\OperatorTok{:}\ErrorTok{=}\StringTok{ }\KeywordTok{exp}\NormalTok{(}\FloatTok{0.1} \OperatorTok{+}\StringTok{ }\NormalTok{yearMinus2000}\OperatorTok{*}\FloatTok{0.1} \OperatorTok{+}\StringTok{ }\NormalTok{seasonalEffect}\OperatorTok{*}\NormalTok{AMPLITUDE)]}
\NormalTok{d[,y}\OperatorTok{:}\ErrorTok{=}\KeywordTok{rpois}\NormalTok{(.N,mu)]}
\end{Highlighting}
\end{Shaded}

\section{Data}\label{data}

Here we show the true data, and note that there is an increasing annual
trend (the data gets higher as time goes on) and there is a seasonal
pattern (one peak/trough per year)

\begin{Shaded}
\begin{Highlighting}[]
\NormalTok{q <-}\StringTok{ }\KeywordTok{ggplot}\NormalTok{(d,}\KeywordTok{aes}\NormalTok{(}\DataTypeTok{x=}\NormalTok{date,}\DataTypeTok{y=}\NormalTok{y))}
\NormalTok{q <-}\StringTok{ }\NormalTok{q }\OperatorTok{+}\StringTok{ }\KeywordTok{geom_line}\NormalTok{(}\DataTypeTok{lwd=}\FloatTok{0.25}\NormalTok{)}
\NormalTok{q <-}\StringTok{ }\NormalTok{q }\OperatorTok{+}\StringTok{ }\KeywordTok{scale_x_date}\NormalTok{(}\StringTok{"Time"}\NormalTok{)}
\NormalTok{q <-}\StringTok{ }\NormalTok{q }\OperatorTok{+}\StringTok{ }\KeywordTok{scale_y_continuous}\NormalTok{(}\StringTok{"Cases"}\NormalTok{)}
\NormalTok{q}
\end{Highlighting}
\end{Shaded}

\includegraphics{website_files/figure-latex/unnamed-chunk-3-1.pdf}

We split out the data for a few years and see a clear seasonal trend:

\begin{Shaded}
\begin{Highlighting}[]
\NormalTok{q <-}\StringTok{ }\KeywordTok{ggplot}\NormalTok{(d[year }\OperatorTok\StringTok{ }\KeywordTok{c}\NormalTok{(}\DecValTok{2005}\OperatorTok{:}\DecValTok{2010}\NormalTok{)],}\KeywordTok{aes}\NormalTok{(}\DataTypeTok{x=}\NormalTok{dayOfYear,}\DataTypeTok{y=}\NormalTok{y))}
\NormalTok{q <-}\StringTok{ }\NormalTok{q }\OperatorTok{+}\StringTok{ }\KeywordTok{facet_wrap}\NormalTok{(}\OperatorTok{~}\NormalTok{year)}
\NormalTok{q <-}\StringTok{ }\NormalTok{q }\OperatorTok{+}\StringTok{ }\KeywordTok{geom_point}\NormalTok{()}
\NormalTok{q <-}\StringTok{ }\NormalTok{q }\OperatorTok{+}\StringTok{ }\KeywordTok{stat_smooth}\NormalTok{(}\DataTypeTok{colour=}\StringTok{"red"}\NormalTok{)}
\NormalTok{q <-}\StringTok{ }\NormalTok{q }\OperatorTok{+}\StringTok{ }\KeywordTok{scale_y_continuous}\NormalTok{(}\StringTok{"Day of year"}\NormalTok{)}
\NormalTok{q <-}\StringTok{ }\NormalTok{q }\OperatorTok{+}\StringTok{ }\KeywordTok{scale_y_continuous}\NormalTok{(}\StringTok{"Cases"}\NormalTok{)}
\end{Highlighting}
\end{Shaded}

\begin{verbatim}
## Scale for 'y' is already present. Adding another scale for 'y', which
## will replace the existing scale.
\end{verbatim}

\begin{Shaded}
\begin{Highlighting}[]
\NormalTok{q}
\end{Highlighting}
\end{Shaded}

\begin{verbatim}
## `geom_smooth()` using method = 'loess' and formula 'y ~ x'
\end{verbatim}

\includegraphics{website_files/figure-latex/unnamed-chunk-4-1.pdf}

\section{Model With Non-Parametric
Seasonality}\label{model-with-non-parametric-seasonality}

If we want to investigate the seasonality of our data, and identify when
are the peaks and troughs, we can use non-parametric approaches. They
are flexible and easy to implement, but they can lack power and be hard
to interpret:

\begin{itemize}
\tightlist
\item
  Create a categorical variable for the seasons (e.g. \texttt{spring},
  \texttt{summer}, \texttt{autumn}, \texttt{winter}) and include this in
  the regression model
\item
  Create a categorical variable for the months (e.g. \texttt{Jan},
  \texttt{Feb}, \ldots{}, \texttt{Dec}) and include this in the
  regression model
\end{itemize}

\begin{Shaded}
\begin{Highlighting}[]
\NormalTok{nfit0 <-}\StringTok{ }\KeywordTok{glm}\NormalTok{(y}\OperatorTok{~}\NormalTok{yearMinus2000 }\OperatorTok{+}\StringTok{ }\NormalTok{dailyrainfall, }\DataTypeTok{data=}\NormalTok{d, }\DataTypeTok{family=}\KeywordTok{poisson}\NormalTok{())}
\NormalTok{nfit1 <-}\StringTok{ }\KeywordTok{glm}\NormalTok{(y}\OperatorTok{~}\NormalTok{yearMinus2000 }\OperatorTok{+}\StringTok{ }\NormalTok{dailyrainfall }\OperatorTok{+}\StringTok{ }\KeywordTok{as.factor}\NormalTok{(month), }\DataTypeTok{data=}\NormalTok{d, }\DataTypeTok{family=}\KeywordTok{poisson}\NormalTok{())}
\end{Highlighting}
\end{Shaded}

We can test the \texttt{month} categorical variable using a likelihood
ratio test:

\begin{Shaded}
\begin{Highlighting}[]
\NormalTok{lmtest}\OperatorTok{::}\KeywordTok{lrtest}\NormalTok{(nfit0, nfit1)}
\end{Highlighting}
\end{Shaded}

\begin{verbatim}
## Likelihood ratio test
## 
## Model 1: y ~ yearMinus2000 + dailyrainfall
## Model 2: y ~ yearMinus2000 + dailyrainfall + as.factor(month)
##   #Df LogLik Df Chisq Pr(>Chisq)    
## 1   3 -26904                        
## 2  14 -13251 11 27307  < 2.2e-16 ***
## ---
## Signif. codes:  0 '***' 0.001 '**' 0.01 '*' 0.05 '.' 0.1 ' ' 1
\end{verbatim}

And then we can look at the output of our regression:

\begin{Shaded}
\begin{Highlighting}[]
\KeywordTok{summary}\NormalTok{(nfit1)}
\end{Highlighting}
\end{Shaded}

\begin{verbatim}
## 
## Call:
## glm(formula = y ~ yearMinus2000 + dailyrainfall + as.factor(month), 
##     family = poisson(), data = d)
## 
## Deviance Residuals: 
##     Min       1Q   Median       3Q      Max  
## -4.2874  -0.9578  -0.1498   0.5894   3.9130  
## 
## Coefficients:
##                     Estimate Std. Error z value Pr(>|z|)    
## (Intercept)        -0.003849   0.028882  -0.133    0.894    
## yearMinus2000       0.101654   0.001053  96.578   <2e-16 ***
## dailyrainfall       0.000442   0.001852   0.239    0.811    
## as.factor(month)2   0.751048   0.029854  25.157   <2e-16 ***
## as.factor(month)3   1.303525   0.027328  47.700   <2e-16 ***
## as.factor(month)4   1.543098   0.026781  57.619   <2e-16 ***
## as.factor(month)5   1.425207   0.026992  52.801   <2e-16 ***
## as.factor(month)6   0.955465   0.028647  33.354   <2e-16 ***
## as.factor(month)7   0.286169   0.032060   8.926   <2e-16 ***
## as.factor(month)8  -0.541443   0.039932 -13.559   <2e-16 ***
## as.factor(month)9  -1.114005   0.049322 -22.586   <2e-16 ***
## as.factor(month)10 -1.350683   0.053389 -25.299   <2e-16 ***
## as.factor(month)11 -1.235671   0.051682 -23.909   <2e-16 ***
## as.factor(month)12 -0.754107   0.042777 -17.629   <2e-16 ***
## ---
## Signif. codes:  0 '***' 0.001 '**' 0.01 '*' 0.05 '.' 0.1 ' ' 1
## 
## (Dispersion parameter for poisson family taken to be 1)
## 
##     Null deviance: 45536.8  on 6939  degrees of freedom
## Residual deviance:  8045.6  on 6926  degrees of freedom
## AIC: 26529
## 
## Number of Fisher Scoring iterations: 5
\end{verbatim}

\emph{NOTE:} See that this is basically the same as a normal regression.

If we want to identify outbreaks, then we need to use the standard
prediction interval formula:

\[
\text{95% CI} = \text{sample average} \pm 1.96 \times \text{sample standard deviation} \sqrt{ 1 + 1 / n}
\] This allows us to identify what the expected thresholds are:

\begin{Shaded}
\begin{Highlighting}[]
\NormalTok{pred <-}\StringTok{ }\KeywordTok{predict}\NormalTok{(nfit1, }\DataTypeTok{type =} \StringTok{"response"}\NormalTok{, }\DataTypeTok{se.fit =}\NormalTok{ T, }\DataTypeTok{newdata =}\NormalTok{ d)}
\NormalTok{d[, threshold0 }\OperatorTok{:}\ErrorTok{=}\StringTok{ }\NormalTok{pred}\OperatorTok{$}\NormalTok{fit]}
\NormalTok{d[, threshold2 }\OperatorTok{:}\ErrorTok{=}\StringTok{ }\KeywordTok{FarringtonThreshold}\NormalTok{(pred, }\DataTypeTok{phi =} \DecValTok{1}\NormalTok{, }\DataTypeTok{z =} \DecValTok{2}\NormalTok{, }\DataTypeTok{skewness.transform =} \StringTok{"2/3"}\NormalTok{)]}
\end{Highlighting}
\end{Shaded}

\begin{Shaded}
\begin{Highlighting}[]
\NormalTok{q <-}\StringTok{ }\KeywordTok{ggplot}\NormalTok{(d[year}\OperatorTok{>}\DecValTok{2015}\NormalTok{],}\KeywordTok{aes}\NormalTok{(}\DataTypeTok{x=}\NormalTok{date,}\DataTypeTok{y=}\NormalTok{y))}
\NormalTok{q <-}\StringTok{ }\NormalTok{q }\OperatorTok{+}\StringTok{ }\KeywordTok{geom_ribbon}\NormalTok{(}\DataTypeTok{mapping=}\KeywordTok{aes}\NormalTok{(}\DataTypeTok{ymin=}\OperatorTok{-}\OtherTok{Inf}\NormalTok{,}\DataTypeTok{ymax=}\NormalTok{threshold2),}\DataTypeTok{fill=}\StringTok{"green"}\NormalTok{,}\DataTypeTok{alpha=}\FloatTok{0.5}\NormalTok{)}
\NormalTok{q <-}\StringTok{ }\NormalTok{q }\OperatorTok{+}\StringTok{ }\KeywordTok{geom_ribbon}\NormalTok{(}\DataTypeTok{mapping=}\KeywordTok{aes}\NormalTok{(}\DataTypeTok{ymin=}\NormalTok{threshold2,}\DataTypeTok{ymax=}\OtherTok{Inf}\NormalTok{),}\DataTypeTok{fill=}\StringTok{"red"}\NormalTok{,}\DataTypeTok{alpha=}\FloatTok{0.5}\NormalTok{)}
\NormalTok{q <-}\StringTok{ }\NormalTok{q }\OperatorTok{+}\StringTok{ }\KeywordTok{geom_line}\NormalTok{(}\DataTypeTok{lwd=}\FloatTok{0.25}\NormalTok{)}
\NormalTok{q <-}\StringTok{ }\NormalTok{q }\OperatorTok{+}\StringTok{ }\KeywordTok{geom_point}\NormalTok{(}\DataTypeTok{data=}\NormalTok{d[year}\OperatorTok{>}\DecValTok{2015} \OperatorTok{&}\StringTok{ }\NormalTok{y}\OperatorTok{>}\NormalTok{threshold2],}\DataTypeTok{colour=}\StringTok{"black"}\NormalTok{,}\DataTypeTok{size=}\FloatTok{2.5}\NormalTok{)}
\NormalTok{q <-}\StringTok{ }\NormalTok{q }\OperatorTok{+}\StringTok{ }\KeywordTok{geom_point}\NormalTok{(}\DataTypeTok{data=}\NormalTok{d[year}\OperatorTok{>}\DecValTok{2015} \OperatorTok{&}\StringTok{ }\NormalTok{y}\OperatorTok{>}\NormalTok{threshold2],}\DataTypeTok{colour=}\StringTok{"red"}\NormalTok{,}\DataTypeTok{size=}\FloatTok{1.5}\NormalTok{)}
\NormalTok{q <-}\StringTok{ }\NormalTok{q }\OperatorTok{+}\StringTok{ }\KeywordTok{scale_x_date}\NormalTok{(}\StringTok{"Time"}\NormalTok{)}
\NormalTok{q <-}\StringTok{ }\NormalTok{q }\OperatorTok{+}\StringTok{ }\KeywordTok{scale_y_continuous}\NormalTok{(}\StringTok{"Cases"}\NormalTok{)}
\NormalTok{q}
\end{Highlighting}
\end{Shaded}

\includegraphics{website_files/figure-latex/unnamed-chunk-9-1.pdf}

\section{Model With Parametric
Seasonality}\label{model-with-parametric-seasonality}

Parametric approaches are more powerful but require more effort:

\begin{itemize}
\tightlist
\item
  Identify the periodicity of the seasonality (how many days between
  peaks?)
\item
  Using trigonometry, transform \texttt{day\ of\ year} into variables
  that appropriately model the observed periodicity
\item
  Obtain coefficient estimates
\item
  Back-transform these estimates into human-understandable values (day
  of peak, day of trough)
\end{itemize}

\emph{NOTE:} You don't always have to investigate seasonality! It
depends entirely on what the purpose of your analysis is!

The Lomb-Scargle Periodogram shows a clear seasonality with a period of
365 days.

\begin{Shaded}
\begin{Highlighting}[]
\CommentTok{# R CODE }
\NormalTok{lomb}\OperatorTok{::}\KeywordTok{lsp}\NormalTok{(d}\OperatorTok{$}\NormalTok{y,}\DataTypeTok{from=}\DecValTok{100}\NormalTok{,}\DataTypeTok{to=}\DecValTok{500}\NormalTok{,}\DataTypeTok{ofac=}\DecValTok{1}\NormalTok{,}\DataTypeTok{type=}\StringTok{"period"}\NormalTok{)}
\end{Highlighting}
\end{Shaded}

\includegraphics{website_files/figure-latex/unnamed-chunk-10-1.pdf}

We then generate two new variables \texttt{cos365} and \texttt{sin365}
and perform a likelihood ratio test to see if they are significant or
not. This is done with two simple poisson regressions.

\begin{Shaded}
\begin{Highlighting}[]
\CommentTok{# R CODE}
\NormalTok{d[,cos365}\OperatorTok{:}\ErrorTok{=}\KeywordTok{cos}\NormalTok{(dayOfYear}\OperatorTok{*}\DecValTok{2}\OperatorTok{*}\NormalTok{pi}\OperatorTok{/}\DecValTok{365}\NormalTok{)]}
\NormalTok{d[,sin365}\OperatorTok{:}\ErrorTok{=}\KeywordTok{sin}\NormalTok{(dayOfYear}\OperatorTok{*}\DecValTok{2}\OperatorTok{*}\NormalTok{pi}\OperatorTok{/}\DecValTok{365}\NormalTok{)]}

\NormalTok{pfit0 <-}\StringTok{ }\KeywordTok{glm}\NormalTok{(y}\OperatorTok{~}\NormalTok{yearMinus2000 }\OperatorTok{+}\StringTok{ }\NormalTok{dailyrainfall, }\DataTypeTok{data=}\NormalTok{d, }\DataTypeTok{family=}\KeywordTok{poisson}\NormalTok{())}
\NormalTok{pfit1 <-}\StringTok{ }\KeywordTok{glm}\NormalTok{(y}\OperatorTok{~}\NormalTok{yearMinus2000 }\OperatorTok{+}\StringTok{ }\NormalTok{dailyrainfall }\OperatorTok{+}\StringTok{ }\NormalTok{sin365 }\OperatorTok{+}\StringTok{ }\NormalTok{cos365, }\DataTypeTok{data=}\NormalTok{d, }\DataTypeTok{family=}\KeywordTok{poisson}\NormalTok{())}
\end{Highlighting}
\end{Shaded}

We can test the seasonality using a likelihood ratio test (which we
already strongly suspected due to the periodogram):

\begin{Shaded}
\begin{Highlighting}[]
\NormalTok{lmtest}\OperatorTok{::}\KeywordTok{lrtest}\NormalTok{(pfit0, pfit1)}
\end{Highlighting}
\end{Shaded}

\begin{verbatim}
## Likelihood ratio test
## 
## Model 1: y ~ yearMinus2000 + dailyrainfall
## Model 2: y ~ yearMinus2000 + dailyrainfall + sin365 + cos365
##   #Df LogLik Df Chisq Pr(>Chisq)    
## 1   3 -26904                        
## 2   5 -12892  2 28024  < 2.2e-16 ***
## ---
## Signif. codes:  0 '***' 0.001 '**' 0.01 '*' 0.05 '.' 0.1 ' ' 1
\end{verbatim}

And then we can look at the output of our regression:

\begin{Shaded}
\begin{Highlighting}[]
\KeywordTok{summary}\NormalTok{(pfit1)}
\end{Highlighting}
\end{Shaded}

\begin{verbatim}
## 
## Call:
## glm(formula = y ~ yearMinus2000 + dailyrainfall + sin365 + cos365, 
##     family = poisson(), data = d)
## 
## Deviance Residuals: 
##     Min       1Q   Median       3Q      Max  
## -4.0676  -0.9229  -0.1170   0.5861   3.4103  
## 
## Coefficients:
##                 Estimate Std. Error z value Pr(>|z|)    
## (Intercept)    0.0887436  0.0176742   5.021 5.14e-07 ***
## yearMinus2000  0.1016117  0.0010525  96.539  < 2e-16 ***
## dailyrainfall  0.0002287  0.0018476   0.124    0.901    
## sin365         1.3972586  0.0103200 135.393  < 2e-16 ***
## cos365        -0.5035265  0.0086308 -58.341  < 2e-16 ***
## ---
## Signif. codes:  0 '***' 0.001 '**' 0.01 '*' 0.05 '.' 0.1 ' ' 1
## 
## (Dispersion parameter for poisson family taken to be 1)
## 
##     Null deviance: 45536.8  on 6939  degrees of freedom
## Residual deviance:  7328.5  on 6935  degrees of freedom
## AIC: 25794
## 
## Number of Fisher Scoring iterations: 5
\end{verbatim}

We also see that the (significant!) coefficient for \texttt{year} is
\texttt{0.1} which means that for each additional year, the outcome
increases by \texttt{exp(0.1)=1.11}. We also see that the coefficient
for \texttt{dailyrainfall} was not significant, which means that we did
not find a significant association between the outcome and
\texttt{dailyrainfall}.

\emph{NOTE:} See that this is basically the same as a normal regression.

Through the likelihood ratio test we saw a clear significant seasonal
effect. We can now use trigonometry to back-calculate the amplitude and
location of peak/troughs from the \texttt{cos365} and \texttt{sin365}
estimates:

\begin{Shaded}
\begin{Highlighting}[]
\NormalTok{b1 <-}\StringTok{ }\FloatTok{1.428417} \CommentTok{# sin coefficient}
\NormalTok{b2 <-}\StringTok{ }\OperatorTok{-}\FloatTok{0.512912} \CommentTok{# cos coefficient}
\NormalTok{amplitude <-}\StringTok{ }\KeywordTok{sqrt}\NormalTok{(b1}\OperatorTok{^}\DecValTok{2} \OperatorTok{+}\StringTok{ }\NormalTok{b2}\OperatorTok{^}\DecValTok{2}\NormalTok{)}
\NormalTok{p <-}\StringTok{ }\KeywordTok{atan}\NormalTok{(b1}\OperatorTok{/}\NormalTok{b2) }\OperatorTok{*}\StringTok{ }\DecValTok{365}\OperatorTok{/}\DecValTok{2}\OperatorTok{/}\NormalTok{pi}
\ControlFlowTok{if}\NormalTok{ (p }\OperatorTok{>}\StringTok{ }\DecValTok{0}\NormalTok{) \{}
\NormalTok{    peak <-}\StringTok{ }\NormalTok{p}
\NormalTok{    trough <-}\StringTok{ }\NormalTok{p }\OperatorTok{+}\StringTok{ }\DecValTok{365}\OperatorTok{/}\DecValTok{2}
\NormalTok{\} }\ControlFlowTok{else}\NormalTok{ \{}
\NormalTok{    peak <-}\StringTok{ }\NormalTok{p }\OperatorTok{+}\StringTok{ }\DecValTok{365}\OperatorTok{/}\DecValTok{2}
\NormalTok{    trough <-}\StringTok{ }\NormalTok{p }\OperatorTok{+}\StringTok{ }\DecValTok{365}
\NormalTok{\}}
\ControlFlowTok{if}\NormalTok{ (b1 }\OperatorTok{<}\StringTok{ }\DecValTok{0}\NormalTok{) \{}
\NormalTok{    g <-}\StringTok{ }\NormalTok{peak}
\NormalTok{    peak <-}\StringTok{ }\NormalTok{trough}
\NormalTok{    trough <-}\StringTok{ }\NormalTok{g}
\NormalTok{\}}
\KeywordTok{print}\NormalTok{(}\KeywordTok{sprintf}\NormalTok{(}\StringTok{"amplitude is estimated as %s, peak is estimated as %s, trough is estimated as %s"}\NormalTok{,}\KeywordTok{round}\NormalTok{(amplitude,}\DecValTok{2}\NormalTok{),}\KeywordTok{round}\NormalTok{(peak),}\KeywordTok{round}\NormalTok{(trough)))}
\end{Highlighting}
\end{Shaded}

\begin{verbatim}
## [1] "amplitude is estimated as 1.52, peak is estimated as 111, trough is estimated as 294"
\end{verbatim}

\begin{Shaded}
\begin{Highlighting}[]
\KeywordTok{print}\NormalTok{(}\KeywordTok{sprintf}\NormalTok{(}\StringTok{"true values are: amplitude: %s, peak: %s, trough: %s"}\NormalTok{,}\KeywordTok{round}\NormalTok{(AMPLITUDE,}\DecValTok{2}\NormalTok{),}\KeywordTok{round}\NormalTok{(}\DecValTok{365}\OperatorTok{/}\DecValTok{4}\OperatorTok{+}\NormalTok{SEASONAL_HORIZONTAL_SHIFT),}\KeywordTok{round}\NormalTok{(}\DecValTok{3}\OperatorTok{*}\DecValTok{365}\OperatorTok{/}\DecValTok{4}\OperatorTok{+}\NormalTok{SEASONAL_HORIZONTAL_SHIFT)))}
\end{Highlighting}
\end{Shaded}

\begin{verbatim}
## [1] "true values are: amplitude: 1.5, peak: 111, trough: 294"
\end{verbatim}

\emph{NOTE:} An amplitude of 1.5 means that when comparing the average
time of year to the peak, the peak is expected to be
\texttt{exp(1.5)=4.5} times higher than average. We take the exponential
because we have run a poisson regression (so think incident rate ratio).

\section{Autocorrelation}\label{autocorrelation-1}

We check the \texttt{pacf} of the residuals to ensure that there is no
autocorrelation. If we observe autocorrelation in our residuals, then we
need to use a \texttt{robust} variance estimator (i.e.~it makes our
estimated variances bigger to account for our poor model fitting).

Here we see that our non-parametric seasonality model has not accounted
for all of the associations in the data, so there is some
autocorrelation in the residuals:

\begin{Shaded}
\begin{Highlighting}[]
\NormalTok{d[,residuals}\OperatorTok{:}\ErrorTok{=}\KeywordTok{residuals}\NormalTok{(nfit1, }\DataTypeTok{type =} \StringTok{"response"}\NormalTok{)]}
\NormalTok{d[,predicted}\OperatorTok{:}\ErrorTok{=}\KeywordTok{predict}\NormalTok{(nfit1, }\DataTypeTok{type =} \StringTok{"response"}\NormalTok{)]}
\KeywordTok{pacf}\NormalTok{(d}\OperatorTok{$}\NormalTok{residuals)}
\end{Highlighting}
\end{Shaded}

\includegraphics{website_files/figure-latex/unnamed-chunk-15-1.pdf}

Here we see that our parametric seasonality model has accounted for all
of the associations in the data, so there is no autocorrelation in the
residuals:

\begin{Shaded}
\begin{Highlighting}[]
\NormalTok{d[,residuals}\OperatorTok{:}\ErrorTok{=}\KeywordTok{residuals}\NormalTok{(pfit1, }\DataTypeTok{type =} \StringTok{"response"}\NormalTok{)]}
\NormalTok{d[,predicted}\OperatorTok{:}\ErrorTok{=}\KeywordTok{predict}\NormalTok{(pfit1, }\DataTypeTok{type =} \StringTok{"response"}\NormalTok{)]}
\KeywordTok{pacf}\NormalTok{(d}\OperatorTok{$}\NormalTok{residuals)}
\end{Highlighting}
\end{Shaded}

\includegraphics{website_files/figure-latex/unnamed-chunk-16-1.pdf}

\bibliography{book.bib,packages.bib}


\end{document}
